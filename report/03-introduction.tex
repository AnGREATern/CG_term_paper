\chapter*{ВВЕДЕНИЕ}
\addcontentsline{toc}{chapter}{ВВЕДЕНИЕ}

Морфинг трехмерных объектов представляет собой технику анимации, при котором объект плавно переходит в другой так, что человеческий глаз не распознает момент, когда заканчивается первый и начинается второй. С помощью морфинга создают плавные анимационные эффекты в различных областях, включая кинопроизводство, рекламу, игры и дизайн. 

Целью данной работы является разработка программы для морфинга объектов. Для достижения поставленной цели требуется решить следующие задачи:

\begin{enumerate}[label={\arabic*)}]
\item проанализировать алгоритмы морфинга трехмерных моделей;
\item выбрать наиболее подходящий для достижения поставленной цели;
\item выбрать средства реализации программы;
\item разработать программу и реализовать выбранные алгоритмы;
\item реализовать графический интерфейс;
\item исследовать временные характеристики выбранного алгоритма морфинга объектов на основе созданной программы.
\end{enumerate}