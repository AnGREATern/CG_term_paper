\chapter*{ВВЕДЕНИЕ}
\addcontentsline{toc}{chapter}{ВВЕДЕНИЕ}

Морфинг трехмерных объектов представляет собой технику анимации, при котором объект плавно переходит в другой так, что человеческий глаз не распознает момент, когда заканчивается первый и начинается второй. С помощью морфинга создают плавные анимационные эффекты в различных областях, включая кинопроизводство, рекламу, игры и дизайн. 

Целью данной работы является разработка приложения
для морфинга объектов. Для достижения поставленной цели
требуется решить следующие задачи:

\begin{enumerate}
\item Проанализировать алгоритмы морфинга трехмерных моделей;
\item Выбрать наиболее подходящий для достижения поставленной цели;
\item Выбрать средства реализации приложения.
\item Разработать приложение и реализовать выбранные алгоритмы.
\item Реализовать графический интерфейс.
\item Исследовать временные характеристики выбранных алгоритмов на основе созданного приложения.
\end{enumerate}