\chapter{Конструкторский раздел}

\section{Требования к программе}

Программа должна предоставлять пользователю следующую функциональность:
\begin{itemize}[label*=---]
	\item загрузка объектов в программу в формате .obj;
	\item выбор цвета загружаемых объектов;
	\item отображение загруженных объектов;
	\item возможность просмотра объектов, путем масштабирования, перемещения и поворота;
	\item перемещение источника света;
	\item морфинг.
\end{itemize}

Программа должна корректно реагировать на любые действия пользователя.

\newpage

\section{Разработка алгоритмов}

\subsection{$Z$-буфер}

На рисунке~\ref{fig:zbuf} приведена схема алгоритма, использующего $z$-буфер.

\begin{figure}[h]
	\centering
	\includesvg[scale=0.75]{inc/img/zbuf}
	\caption{Схема алгоритма, использующего $z$-буфер}
	\label{fig:zbuf}
\end{figure}

\clearpage

\subsection{Закраска по методу Гуро}

На рисунке~\ref{fig:guro} приведена схема алгоритма закраски по методу Гуро.

\begin{figure}[h]
	\centering
	\includesvg[scale=0.75]{inc/img/guro}
	\caption{схема алгоритма закраски по методу Гуро}
	\label{fig:guro}
\end{figure}

\clearpage

\subsection{Морфинг слиянием}

На рисунках~\ref{fig:morph1},~\ref{fig:morph2}~и~\ref{fig:morph2} приведена схема алгоритма морфинга слиянием.

\begin{figure}[h]
	\centering
	\includesvg[scale=0.7]{inc/img/morph1}
	\caption{схема алгоритма морфинга слиянием, часть 1}
	\label{fig:morph1}
\end{figure}

\begin{figure}[h]
	\centering
	\includesvg[scale=0.75]{inc/img/morph2}
	\caption{схема алгоритма морфинга слиянием, часть 2}
	\label{fig:morph2}
\end{figure}

\clearpage

\begin{figure}[h]
	\centering
	\includesvg[scale=0.75]{inc/img/morph3}
	\caption{схема алгоритма морфинга слиянием, часть 3}
	\label{fig:morph3}
\end{figure}

\section*{Вывод}
Были описаны алгоритмы, использующего $z$-буфер, закраски по Гуро и морфинга слиянием. Были выдвинуты требования к программе.