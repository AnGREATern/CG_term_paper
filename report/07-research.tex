\chapter{Исследовательский раздел}

\section{Технические характеристики}

Технические характеристики используемого оборудования:

\begin{itemize}[label=---]
	\item операционная система: Ubuntu 22.04.4 LTS;
	\item оперативная память: 16 ГБ;
	\item процессор: Intel(R) Core(TM) i5-10300H.
\end{itemize}

\section{Исследование влияния количества ребер на время выполнения морфинга}

Для проведения временных замеров использовалась библиотека $Criterion$. Количество замеров для каждого набора входных данных --- 100.

Так как время выполнения алгоритма зависит от количества ребер и пересечений их проекций на сфере, замеры проводились для двух случаев:
\begin{enumerate}[label={\arabic*)}]
	\item стартовый и конечный объекты имеют одинаковую топологию;
	\item стартовый и конечный объекты имеют разную топологию.
\end{enumerate}

\begin{table}[ht]
	\small
	\begin{center}
		\begin{threeparttable}
			\caption{Результаты временных замеров алгоритма морфинга слиянием}
			\label{tbl:res1}
			\begin{tabular}{|r|r|r|}
				\hline
				\bfseries Количество ребер & \multicolumn{2}{c|}{\bfseries Время, с} \\
				\cline{2-3}
				& \bfseries Одинаковые объекты & \bfseries Разные объекты \\
				\hline
				408 & 0.001 & 0.005 \\
				\hline
				1338 & 0.008 & 0.236 \\
				\hline
				2268 & 0.017 & 0.489 \\
				\hline
				3198 & 0.024 & 0.707 \\
				\hline
				4128 & 0.036 & 0.868 \\
				\hline
				5368 & 0.206 & 1.342 \\
				\hline
				6608 & 0.341 & 1.670 \\
				\hline
				7848 & 0.557 & 2.298 \\
				\hline
			\end{tabular}
		\end{threeparttable}
	\end{center}
\end{table}


На рисунке~\ref{img:cpu_times} представлены результаты временных замеров алгоритма морфинга слиянием.

\begin{figure}[h]
	\centering
	\includesvg{inc/img/graph_time}
	\caption{Результаты временных замеров алгоритма морфинга слиянием}
	\label{img:cpu_times}
\end{figure}

\pagebreak

С увеличением количества ребер и количества пересечений их проекций, время выполнения растет. На 7848 ребрах морфинг с нулем пересечений ребер (с одинаковыми объектами) выполнялся в среднем за 0.557 с, а с некоторым ненулевым числом пересечений за 2.298 с, то есть примерно в 4.13 раз медленнее.

На 1338 ребрах морфинг с разными объектами выполнялся 0.236 с, а на 7848 --- 2.298 с, то есть время выполнения растет быстрее, чем количество вершин.

\begin{table}[h!]
	\small
	\caption{Результаты временных замеров алгоритма морфинга слиянием}
	\label{tbl:res2}
	\begin{center}
		\begin{tabular}{|rr|rr|}
			\hline
			\multicolumn{2}{|c|}{\textbf{Количество ребер}}             & \multicolumn{2}{c|}{\textbf{Время, с}}                              \\ \hline
			\multicolumn{1}{|r|}{\textbf{Объект А}} & \textbf{Объект Б} & \multicolumn{1}{r|}{\textbf{Из А в Б}} & \textbf{Из Б в А}          \\ \hline
			\multicolumn{1}{|r|}{36}                & 372               & \multicolumn{1}{r|}{0.005}             & \multicolumn{1}{r|}{0.005} \\ \hline
			\multicolumn{1}{|r|}{36}                & 7476              & \multicolumn{1}{r|}{2.050}             & \multicolumn{1}{r|}{2.050} \\ \hline
			\multicolumn{1}{|r|}{372}               & 7476              & \multicolumn{1}{r|}{2.298}                  & \multicolumn{1}{r|}{2.298}                  \\ \hline
			\multicolumn{1}{|r|}{36}                & 73728             & \multicolumn{1}{r|}{59.641}                  &         59.641                   \\ \hline
		\end{tabular}
	\end{center}
\end{table}

В соответствии с таблицей~\ref{tbl:res2}, скорость работы алгоритма морфинга слиянием не зависит от перестановки объектов местами.

\section*{Вывод}

В данном разделе были проведены исследования по замеру времени морфинга объектов в зависимости от наличия пересечений ребер и количества их, а также установлена независимость скорость работы алгоритма от перестановки объектов.
