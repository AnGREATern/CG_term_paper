\chapter{Технологический раздел}

\section{Средства реализации}

В качестве языка программирования был выбран $Rust$. Данный выбор был обусловлен следующими факторами:

\begin{itemize}[label*=---]
	\item средствами языка можно реализовать все алгоритмы, выбранные в результате проектирования;
	\item $Rust$ обладает высокой производительностью. \cite{RUST_IS_THE_BEST}
\end{itemize}

% \clearpage

\section{Реализация алгоритмов}

В листингах~\ref{lst:zbuf1.txt},~\ref{lst:zbuf2.txt}~и~\ref{lst:zbuf3.txt} приведена реализация алгоритма $z$-буфера, совмещенного с закраской по Гуро. 

\includelisting
{zbuf1.txt}
{Реализация алгоритма $z$-буфера, совмещенного с закраской по Гуро, часть 1}

\clearpage

\includelisting
{zbuf2.txt}
{Реализация алгоритма $z$-буфера, совмещенного с закраской по Гуро, часть 2}

\clearpage

\includelisting
{zbuf3.txt}
{Реализация алгоритма $z$-буфера, совмещенного с закраской по Гуро, часть 3}

\clearpage

В листингах~\ref{lst:morph1.txt},~\ref{lst:morph2.txt},~\ref{lst:morph3.txt},~\ref{lst:morph4.txt}~и~\ref{lst:morph5.txt} приведена реализация алгоритма морфинга слиянием. 

\includelisting
{morph1.txt}
{Реализация алгоритма морфинга слиянием, часть 1}

\clearpage

\includelisting
{morph2.txt}
{Реализация алгоритма морфинга слиянием, часть 2}

\clearpage

\includelisting
{morph3.txt}
{Реализация алгоритма морфинга слиянием, часть 3}

\clearpage

\includelisting
{morph4.txt}
{Реализация алгоритма морфинга слиянием, часть 4}

\clearpage

\includelisting
{morph5.txt}
{Реализация алгоритма морфинга слиянием, часть 5}

\clearpage

\section{Интерфейс программы}

На рисунках \ref{img:main} и \ref{img:menus} представлен интерфейс программы.

\includeimage
{main}
{f}
{!ht}
{0.8\textwidth}
{Общий вид программы}

\includeimage
{menus}
{f}
{!ht}
{0.8\textwidth}
{Внутренние меню программы}

Во вкладке <<Загрузить объекты>> пользователь загружает 2 объекта в формате $.obj$, цвета которых предварительно выбираются во вкладке <<Выбор цвета>>, а также имеет возможность поменять местами стартовый и итоговый объекты.

Во вкладке <<Обозреваемый объект>> выбирается объект для его просмотра и преобразования: перемещения, поворота и масштабирования.

\section*{Вывод}
Программа была реализована.


